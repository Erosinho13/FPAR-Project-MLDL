\documentclass[10pt,twocolumn,hidelinks,letterpaper]{article}

\usepackage[table]{xcolor}
\usepackage{cvpr}
\usepackage{times}
\usepackage{epsfig}
\usepackage{graphicx}
\usepackage{amsmath}
\usepackage{amssymb}

\usepackage[utf8x]{inputenc}
\usepackage[english]{babel}
\usepackage{url}
\usepackage{lipsum}
\usepackage{color}

\usepackage[framed , numbered]{matlab-prettifier}
\usepackage{listings}
\usepackage{setspace}
\usepackage{color} %red, green, blue, yellow, cyan, magenta, black, white
\usepackage{geometry}
\usepackage{subcaption}
\usepackage{graphicx}
\usepackage{float}


% Include other packages here, before hyperref.

% If you comment hyperref and then uncomment it, you should delete
% egpaper.aux before re-running latex.  (Or just hit 'q' on the first latex
% run, let it finish, and you should be clear).
\usepackage[breaklinks=true,bookmarks=false]{hyperref}

\cvprfinalcopy % *** Uncomment this line for the final submission

\def\cvprPaperID{****} % *** Enter the CVPR Paper ID here
\def\httilde{\mbox{\tt\raisebox{-.5ex}{\symbol{126}}}}

% Pages are numbered in submission mode, and unnumbered in camera-ready
%\ifcvprfinal\pagestyle{empty}\fi
\setcounter{page}{1}
\begin{document}

%%%%%%%%% TITLE
\title{01TXFSM - Machine Learning and Deep Learning \\
\vspace{0.3in}
Homework 3 \\
Deep Domain Adaptation}

\author{Eros Fanì - s269781\\
Politecnico di Torino\\
{\tt\small eros.fani@studenti.polito.it}
\and
Gabriele Trivigno - s276807\\
Politecnico di Torino\\
{\tt\small gabriele.trivigno@studenti.polito.it}
\and
Cristiano Gerbino - s277058\\
Politecnico di Torino\\
{\tt\small s277058@studenti.polito.it}
}

\maketitle
%\thispagestyle{empty}

\lstset{language=python,%
  %basicstyle=\color{red},
  breaklines=true,%
  morekeywords={matlab2tikz},
  keywordstyle={\small \color{blue}},%
  morekeywords=[2]{1}, keywordstyle=[2]{\small \color{black}},
  identifierstyle={\small \color{black}},%
  stringstyle={\small \color{mylilas}},
  commentstyle={\small \color{mygreen}},%
  showstringspaces=false,%without this there will be a symbol in the places where there is a space
  numbers=left,%
  numberstyle={\small \color{black}},% size of the numbers
  numbersep=7pt, % this defines how far the numbers are from the text
  emph=[1]{error,warning},emphstyle=[1]{\small \color{red}} %some words to emphasise
  %emph=[2]{word1,word2}, emphstyle=[2]{style},
}

%%%%%%%%% BODY TEXT
\newcommand{\quotes}[1]{“#1”}
\newcommand{\reff}[1]{Figure \ref{#1}}
\setlength{\parindent}{0pt}
\setstretch{0.1}
\setlength{\parskip}{1em}
\definecolor{TopRow}{HTML}{E6E6FF}

\begin{abstract}

\end{abstract}

\section{Introduction}


{\small
\bibliographystyle{ieee}
\bibliography{egbib}
}

\end{document}
